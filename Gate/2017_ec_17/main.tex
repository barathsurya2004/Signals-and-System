\documentclass[12pt,-letter paper]{article}
\usepackage{siunitx}
\usepackage{setspace}
\usepackage{gensymb}
\usepackage{xcolor}
\usepackage{caption}
%\usepackage{subcaption}
\doublespacing
\singlespacing
\usepackage[none]{hyphenat}
\usepackage{amssymb}
\usepackage{relsize}
\usepackage[cmex10]{amsmath}
\usepackage{mathtools}
\usepackage{amsmath}
\usepackage{commath}
\usepackage{amsthm}
\usepackage{graphicx} 
% \usepackage{circuitikz} 
\interdisplaylinepenalty=2500
%\savesymbol{iint}
\usepackage{txfonts}
%\restoresymbol{TXF}{iint}
\usepackage{wasysym}
\usepackage{amsthm}
\usepackage{mathrsfs}
\usepackage{txfonts}
\let\vec\mathbf{}
\usepackage{stfloats}
\usepackage{float}
\usepackage{hyperref}
\usepackage{cite}
\usepackage{cases}
\usepackage{subfig}
%\usepackage{xtab}
\usepackage{longtable}
\usepackage{multirow}
%\usepackage{algorithm}
\usepackage{amssymb}
%\usepackage{algpseudocode}
\usepackage{enumitem}
\usepackage{mathtools}
%\usepackage{eenrc}
\usepackage[framemethod=tikz]{mdframed}
\usepackage{listings}
%\usepackage{listings}
\usepackage[latin1]{inputenc}
%%\usepackage{color}{   
%%\usepackage{lscape}
\usepackage{textcomp}
\usepackage{titling}
\usepackage{hyperref}
% \usepackage{fulbigskip}   
\usepackage{tikz}
% \usepackage{graphicx}
%\usepackage[left=1in, right=2in, top=1in, bottom=1in]{geometry}

\let\vec\mathbf{}
\usepackage{enumitem}
\usepackage{graphicx}
\usepackage{siunitx}
\let\vec\mathbf{}
\usepackage{enumitem}
\usepackage{graphicx}
\usepackage{enumitem}
\usepackage{tfrupee}
\usepackage{amsmath}
\usepackage{amssymb}
\usepackage{tfrupee}
\DeclareMathOperator*{\Res}{Res}
\newtheorem{theorem}{Theorem}[section]
\newtheorem{problem}{Problem}
\newtheorem{proposition}{Proposition}[section]
\newtheorem{lemma}{Lemma}[section]
\newtheorem{corollary}[theorem]{Corollary}
\newtheorem{example}{Example}[section]
\newtheorem{definition}[problem]{Definition}
\newcommand{\BEQA}{\begin{eqnarray}}
\newcommand{\EEQA}{\end{eqnarray}}
\newcommand{\define}{\stackrel{\triangle}{=}}
\theoremstyle{remark}
\newtheorem{rem}{Remark}

\renewcommand{\thefigure}{\theenumi}
\renewcommand{\thetable}{\theenumi}
\providecommand{\pr}[1]{\ensuremath{\Pr\left(#1\right)}}
\providecommand{\prt}[2]{\ensuremath{p_{#1}^{\left(#2\right)} }}        % own macro for this question
\providecommand{\qfunc}[1]{\ensuremath{Q\left(#1\right)}}
\providecommand{\sbrak}[1]{\ensuremath{{}\left[#1\right]}}
\providecommand{\lsbrak}[1]{\ensuremath{{}\left[#1\right.}}
\providecommand{\rsbrak}[1]{\ensuremath{{}\left.#1\right]}}
\providecommand{\brak}[1]{\ensuremath{\left(#1\right)}}
\providecommand{\lbrak}[1]{\ensuremath{\left(#1\right.}}
\providecommand{\rbrak}[1]{\ensuremath{\left.#1\right)}}
\providecommand{\cbrak}[1]{\ensuremath{\left\{#1\right\}}}
\providecommand{\lcbrak}[1]{\ensuremath{\left\{#1\right.}}
\providecommand{\rcbrak}[1]{\ensuremath{\left.#1\right\}}}
\newcommand{\sgn}{\mathop{\mathrm{sgn}}}
\providecommand{\abs}[1]{\left\vert#1\right\vert}
\providecommand{\res}[1]{\Res\displaylimits_{#1}} 
\providecommand{\norm}[1]{\left\lVert#1\right\rVert}
%\providecommand{\norm}[1]{\lVert#1\rVert}
\providecommand{\mtx}[1]{\mathbf{#1}}
\providecommand{\mean}[1]{E\left[ #1 \right]}
\providecommand{\cond}[2]{#1\middle|#2}
\providecommand{\fourier}{\overset{\mathcal{F}}{ \rightleftharpoons}}
%\providecommand{\hilbert}{\overset{\mathcal{H}}{ \rightleftharpoons}}
%\providecommand{\system}{\overset{\mathcal{H}}{ \longleftrightarrow}}
	%\newcommand{\solution}[2]{\textbf{Solution:}{#1}}
\newcommand{\solution}{\noindent \textbf{Solution: }}
\newcommand{\cosec}{\,\text{cosec}\,}
\providecommand{\dec}[2]{\ensuremath{\overset{#1}{\underset{#2}{\gtrless}}}}
\newcommand{\myvec}[1]{\ensuremath{\begin{pmatrix}#1\end{pmatrix}}}
\newcommand{\mydet}[1]{\ensuremath{\begin{vmatrix}#1\end{vmatrix}}}
\providecommand{\rank}{\text{rank}}
\providecommand{\pr}[1]{\ensuremath{\Pr\left(#1\right)}}
\providecommand{\qfunc}[1]{\ensuremath{Q\left(#1\right)}}
	\newcommand*{\permcomb}[4][0mu]{{{}^{#3}\mkern#1#2_{#4}}}
\newcommand*{\perm}[1][-3mu]{\permcomb[#1]{P}}
\newcommand*{\comb}[1][-1mu]{\permcomb[#1]{C}}
\providecommand{\qfunc}[1]{\ensuremath{Q\left(#1\right)}}
\providecommand{\gauss}[2]{\mathcal{N}\ensuremath{\left(#1,#2\right)}}
\providecommand{\diff}[2]{\ensuremath{\dfrac{d{#1}}{d{#2}}}}
\providecommand{\myceil}[1]{\left \lceil #1 \right \rceil }
\newcommand{\sinc}{\,\text{sinc}\,}
\newcommand{\rect}{\,\text{rect}\,}
\newcommand{\E}{\mathbb{E}}
\newcommand{\Var}{\mathrm{Var}}


\begin{document}
\vspace{3cm}

\title{Assignment}
\author{FWC22245 - Barath Surya M}
\maketitle
Consider the $D$-Latch shown in the figure, which is transparent when its clock input $CK$ is high and has zero propagation delay. In the figure, the clock signal $CLK1$ has $50\%$ duty cycle and $CLK2$ is a one fifth period delayed version of $CLK1$. The duty cycle at the output of the latch in percentage is.
\begin{figure}[h!]
	\begin{center}
		
\begin{tikzpicture}
    \draw[draw=black, thin, solid] (-5.00,0.00) -- (-4.50,0.00);
    \draw[draw=black, thin, solid] (-4.50,0.00) -- (-4.50,0.50);
    \draw[draw=black, thin, solid] (-4.50,0.50) -- (-3.50,0.50);
    \draw[draw=black, thin, solid] (-3.50,0.50) -- (-3.50,0.00);
    \draw[draw=black, thin, solid] (-3.50,0.00) -- (-3.00,0.00);
    \draw[draw=black, thin, solid] (-3.00,0.00) -- (-2.50,0.00);
    \draw[draw=black, thin, solid] (-2.50,0.00) -- (-2.50,0.50);
    \draw[draw=black, thin, solid] (-2.50,0.50) -- (-1.50,0.50);
    \draw[draw=black, thin, solid] (-1.50,0.50) -- (-1.50,0.00);
    \draw[draw=black, thin, solid] (-1.50,0.00) -- (-0.50,0.00);
    \draw[draw=black, thin, solid] (-0.50,0.00) -- (-0.50,0.50);
    \draw[draw=black, thin, solid] (-0.50,0.50) -- (0.00,0.50);
    \draw[draw=black, thin, solid] (-5.00,-1.00) -- (-4.00,-1.00);
    \draw[draw=black, thin, solid] (-4.00,-1.00) -- (-4.00,-0.50);
    \draw[draw=black, thin, solid] (-4.00,-0.50) -- (-3.00,-0.50);
    \draw[draw=black, thin, solid] (-3.00,-0.50) -- (-3.00,-1.00);
    \draw[draw=black, thin, solid] (-3.00,-1.00) -- (-2.00,-1.00);
    \draw[draw=black, thin, solid] (-2.00,-1.00) -- (-2.00,-0.50);
    \draw[draw=black, thin, solid] (-2.00,-0.50) -- (-1.00,-0.50);
    \draw[draw=black, thin, solid] (-1.00,-0.50) -- (-1.00,-1.00);
    \draw[draw=black, thin, solid] (-1.00,-1.00) -- (0.00,-1.00);
    \draw[draw=black, thin, dotted] (-4.50,0.00) -- (-4.50,-2.00);
    \draw[draw=black, thin, dotted] (-4.00,0.50) -- (-4.00,-2.00);
    \draw[draw=black, -latex, thin, solid] (-5.00,-2.00) -- (-4.50,-2.00);
    \draw[draw=black, -latex, thin, solid] (-3.50,-2.00) -- (-4.00,-2.00);
    \draw[draw=black, latex-, thin, solid] (-4.50,1.00) -- (-3.50,1.00);
    \draw[draw=black, latex-, thin, solid] (-2.50,1.00) -- (-3.00,1.00);
    \draw[draw=black, thin, solid] (1.50,1.00) -- (1.50,-1.00);
    \draw[draw=black, thin, solid] (1.50,-1.00) -- (3.50,-1.00);
    \draw[draw=black, thin, solid] (3.50,1.00) -- (3.50,-1.00);
    \draw[draw=black, thin, solid] (1.50,1.00) -- (3.50,1.00);
    \draw[draw=black, thin, solid] (2.50,-1.00) -- (2.50,-1.50);
    \draw[draw=black, thin, solid] (2.50,-1.50) -- (1.00,-1.50);
    \draw[draw=black, thin, solid] (1.50,0.50) -- (0.50,0.50);
    \draw[draw=black, thin, solid] (3.50,0.50) -- (4.50,0.50);
    \draw[draw=black, thin, solid] (0.50,0.50) circle (0.1);
    \draw[draw=black, thin, solid] (4.50,0.50) circle (0.1);
    \draw[draw=black, thin, solid] (1.00,-1.50) circle (0.1);
    \node[black, anchor=south west] at (-6,0){\footnotesize $CLK1$};
    \node[black, anchor=south west] at (-6,-1) {\footnotesize $CLK2$};
    \node[black, anchor=south west] at (-0.06,0.75){\footnotesize $CLK1$};
    \node[black, anchor=south west] at (-0.06,-2.25) {\footnotesize $CLK2$};
    \node[black, anchor=south west] at (3.94,0.75) {Output};
    \node[black, anchor=south west] at (1.54,0.25) {\footnotesize $D$};
    \node[black, anchor=south west] at (2.84,0.21) {\footnotesize $Q$};
    \node[black, anchor=south west] at (1.80,-0.25) {\footnotesize $D$-latch};
    \node[black, anchor=south west] at (2.05,-1.0) {\footnotesize $CK$};
    \node[black, anchor=south west] at (-4.76,-3) {\footnotesize $\dfrac{T_{CLK}}{5}$};
    \node[black, anchor=south west] at (-3.8,1.1) {\footnotesize $T_{CLK}$};
\end{tikzpicture}
		\caption{$D$-latch and Clock}
	\end{center}
\end{figure}

\end{document}
