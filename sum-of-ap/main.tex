\let\negmedspace\undefined
\let\negthickspace\undefined
\documentclass[journal,12pt,onecolumn]{IEEEtran}
\usepackage{cite}
\usepackage{amsmath,amssymb,amsfonts,amsthm}
\usepackage{algorithmic}
\usepackage{graphicx}
\usepackage{textcomp}
\usepackage{xcolor}
\usepackage{txfonts}
\usepackage{listings}
\usepackage{enumitem}
\usepackage{mathtools}
\usepackage{gensymb}
\usepackage{comment}
\usepackage[breaklinks=true]{hyperref}
\usepackage{tkz-euclide}
\usepackage{listings}

\DeclareMathOperator*{\Res}{Res}
\newtheorem{theorem}{Theorem}[section]
\newtheorem{problem}{Problem}
\newtheorem{proposition}{Proposition}[section]
\newtheorem{lemma}{Lemma}[section]
\newtheorem{corollary}[theorem]{Corollary}
\newtheorem{example}{Example}[section]
\newtheorem{definition}[problem]{Definition}
\newcommand{\BEQA}{\begin{eqnarray}}
\newcommand{\EEQA}{\end{eqnarray}}
\newcommand{\define}{\stackrel{\triangle}{=}}
\theoremstyle{remark}
\newtheorem{rem}{Remark}

\renewcommand{\thefigure}{\theenumi}
\renewcommand{\thetable}{\theenumi}
\renewcommand\thesection{\arabic{section}}
\renewcommand\thesubsection{\thesection.\arabic{subsection}}
\renewcommand\thesubsubsection{\thesubsection.\arabic{subsubsection}}

\renewcommand\thesectiondis{\arabic{section}}
\renewcommand\thesubsectiondis{\thesectiondis.\arabic{subsection}}
\renewcommand\thesubsubsectiondis{\thesubsectiondis.\arabic{subsubsection}}
\providecommand{\pr}[1]{\ensuremath{\Pr\left(#1\right)}}
\providecommand{\prt}[2]{\ensuremath{p_{#1}^{\left(#2\right)} }}        % own macro for this question
\providecommand{\qfunc}[1]{\ensuremath{Q\left(#1\right)}}
\providecommand{\sbrak}[1]{\ensuremath{{}\left[#1\right]}}
\providecommand{\lsbrak}[1]{\ensuremath{{}\left[#1\right.}}
\providecommand{\rsbrak}[1]{\ensuremath{{}\left.#1\right]}}
\providecommand{\brak}[1]{\ensuremath{\left(#1\right)}}
\providecommand{\lbrak}[1]{\ensuremath{\left(#1\right.}}
\providecommand{\rbrak}[1]{\ensuremath{\left.#1\right)}}
\providecommand{\cbrak}[1]{\ensuremath{\left\{#1\right\}}}
\providecommand{\lcbrak}[1]{\ensuremath{\left\{#1\right.}}
\providecommand{\rcbrak}[1]{\ensuremath{\left.#1\right\}}}
\newcommand{\sgn}{\mathop{\mathrm{sgn}}}
\providecommand{\abs}[1]{\left\vert#1\right\vert}
\providecommand{\res}[1]{\Res\displaylimits_{#1}} 
\providecommand{\norm}[1]{\left\lVert#1\right\rVert}
%\providecommand{\norm}[1]{\lVert#1\rVert}
\providecommand{\mtx}[1]{\mathbf{#1}}
\providecommand{\mean}[1]{E\left[ #1 \right]}
\providecommand{\cond}[2]{#1\middle|#2}
\providecommand{\fourier}{\overset{\mathcal{F}}{ \rightleftharpoons}}
%\providecommand{\hilbert}{\overset{\mathcal{H}}{ \rightleftharpoons}}
%\providecommand{\system}{\overset{\mathcal{H}}{ \longleftrightarrow}}
	%\newcommand{\solution}[2]{\textbf{Solution:}{#1}}
\newcommand{\solution}{\noindent \textbf{Solution: }}
\newcommand{\cosec}{\,\text{cosec}\,}
\providecommand{\dec}[2]{\ensuremath{\overset{#1}{\underset{#2}{\gtrless}}}}
\newcommand{\myvec}[1]{\ensuremath{\begin{pmatrix}#1\end{pmatrix}}}
\newcommand{\mydet}[1]{\ensuremath{\begin{vmatrix}#1\end{vmatrix}}}
\providecommand{\rank}{\text{rank}}
\providecommand{\pr}[1]{\ensuremath{\Pr\left(#1\right)}}
\providecommand{\qfunc}[1]{\ensuremath{Q\left(#1\right)}}
	\newcommand*{\permcomb}[4][0mu]{{{}^{#3}\mkern#1#2_{#4}}}
\newcommand*{\perm}[1][-3mu]{\permcomb[#1]{P}}
\newcommand*{\comb}[1][-1mu]{\permcomb[#1]{C}}
\providecommand{\qfunc}[1]{\ensuremath{Q\left(#1\right)}}
\providecommand{\gauss}[2]{\mathcal{N}\ensuremath{\left(#1,#2\right)}}
\providecommand{\diff}[2]{\ensuremath{\frac{d{#1}}{d{#2}}}}
\providecommand{\myceil}[1]{\left \lceil #1 \right \rceil }
\newcommand\figref{Fig.~\ref}
\newcommand\tabref{Table~\ref}
\newcommand{\sinc}{\,\text{sinc}\,}
\newcommand{\rect}{\,\text{rect}\,}
\newcommand{\E}{\mathbb{E}}
\newcommand{\Var}{\mathrm{Var}}

\begin{document}

\bibliographystyle{IEEEtran}
\vspace{3cm}

\title{EE1205}
\author{EE22BTECH11014 - Barath Surya M}
\maketitle
Q2) The sum of the third and the seventh terms of AP is 6 and their product is 8. Find the sum of first sixteen terms of the AP\\
\solution
The general Term of an ap is
\begin{align}
    x\brak{n} & = x\brak{0} + nd
\end{align}
from the given values
\begin{align}
    x\brak{2} & = x\brak{0} + 2d \label{eq:sumofapgiven1} \\
    x\brak{6} & = x\brak{0} +6d \label{eq:sumofapgiven2}  \\
    x\brak{2} + x\brak{6} = x\brak{2} + x\brak{6} \label{eq:sumofapgiven3}
\end{align}
on solving \eqref{eq:sumofapgiven3} using \eqref{eq:sumofapgiven1} and \eqref{eq:sumofapgiven2}
we get
\begin{align}
    x\brak{6} & =2 \text{ or } 4  \\
    x\brak{2} & = 4 \text{ or } 2
\end{align}
then when $x\brak{6}=2$ and $x\brak{2}=4$
\begin{align}
    x\brak{0} & =5             \\
    d         & = \frac{-1}{2}
\end{align}
then when $x\brak{6}=4$ and $x\brak{2}=2$
\begin{align}
    x\brak{0} & =1            \\
    d         & = \frac{1}{2}
\end{align}

Sum of ap till n terms is
\begin{align}
    s\brak{n} & =\sum_{0}^{n} x\brak{k}                          \\
              & = \sum_{k=-\infty}^{\infty} x\brak{k}u\brak{n-k} \\
              & = x\brak{n} * u\brak{n}
\end{align}
Taking Z transform,
\begin{align}
    S\brak{z} & = X\brak{z}U\brak{z}
\end{align}
from the z-transformation of $x\brak{n}$
\begin{align}
    S\brak{z} & = \brak{\frac{x\brak{0}}{1-z^{-1}} + \frac{dz^{-1}}{\brak{1-z^{-1}}^2}}\brak{\frac{1}{1-z^{-1}}} \quad \abs{z} >\abs{1}     \\
              & = \frac{x\brak{0}}{\brak{1-z^{-1}}^2}+ d\frac{z^{-1}}{\brak{1-z^{-1}}^3} \quad \abs{z} > \abs{1} \label{eq:ztransformofSum}
\end{align}
taking inverse z-transform using contour integration
\begin{align}
    s\brak{n} & = \frac{1}{2\pi j} \oint_C S\brak{z} z^{n-1} dz
\end{align}
where $C$ is clockwise closed contour in region of convergence of $S\brak{z} $.\\
Contour integrals of this form can be solved using Cauchy's residue theorem, which states
\begin{align}
    s\brak{n} & = \frac{1}{2\pi j} \oint_C S\brak{z} z^{n-1} dz                                    \\
              & = \sum \sbrak{\text{residues of } S\brak{z}z^{n-1} \text{ at the poles inside } C}
\end{align}
Residue of a mth order pole is
\begin{align}
    X\brak{z}z^{n-1}                              & = \frac{X'\brak{z}}{\brak{z-z_0}^m}                                      \\
    Res\sbrak{X\brak{z}z^{n-1} \text{ at } z=z_0} & = \frac{1}{\brak{m-1}!} \brak{\frac{d^{m-1}\brak{X'\brak{z}}}{dz^{m-1}}}
\end{align}
for simple first order pole
\begin{align}
    Res\sbrak{X\brak{z}z^{n-1} \text{ at } z=z_0} & = X'\brak{z_0}
\end{align}
So for \eqref{eq:ztransformofSum}
\begin{align}
    S\brak{z}z^{n-1} & = \frac{x\brak{0}z^{n+1}\brak{z-1} + d z^{n+1}}{\brak{z-1}^3}                                        \\
    s\brak{n}        & =\frac{1}{2} \brak{\frac{d^{2}\brak{S\brak{z}\brak{z}^{n-1}}}{dz^2}} \quad \text{at } z=1            \\
                     & = \frac{1}{2} \frac{d^2\brak{x\brak{0}z^{n+2}-x\brak{0}z^{n+1}+dz^{n+1}}}{dz^2}\quad \text{at } z=1  \\
                     & =\frac{1}{2}\brak{\brak{n+1}z^{n-1}\brak{\brak{n+2}x\brak{0}z -nx\brak{0} +dn}} \quad \text{at } z=1 \\
                     & = \frac{1}{2} \brak{\brak{n+1}\brak{\brak{n+2}x\brak{0} -nx\brak{0} +dn}}                            \\
                     & = \frac{n+1}{2}\brak{2x\brak{0} +dn}
\end{align}
Sum of first 16 terms if $x\brak{6} = 2$ and $x\brak{2} = 4$
\begin{align}
    S\brak{15} & = \frac{16}{2} \brak{2\brak{5} + 15\brak{\frac{-1}{2}}} \\
               & =20
\end{align}
Sum of first 16 terms if $x\brak{6} =4$ and $x\brak{2}=2$
\begin{align}
    s\brak{15} & = \frac{16}{2}\brak{2\brak{1} + 15 \brak{\frac{1}{2}}} \\
               & =76
\end{align}
\end{document}
